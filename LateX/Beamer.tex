%%%%%%%%%%%%%%%%%%%%%%%%%%%%%%%%%%%%%%%%%
% Beamer Presentation
% LaTeX Template
% Version 1.0 (10/11/12)
%
% This template has been downloaded from:
% http://www.LaTeXTemplates.com
%
% License:
% CC BY-NC-SA 3.0 (http://creativecommons.org/licenses/by-nc-sa/3.0/)
%
%%%%%%%%%%%%%%%%%%%%%%%%%%%%%%%%%%%%%%%%%

%----------------------------------------------------------------------------------------
%	PACKAGES AND THEMES
%----------------------------------------------------------------------------------------

\documentclass{beamer}

\mode<presentation> {
  \usetheme{Madrid}       % or try default, Darmstadt, Warsaw, ...
  \usecolortheme{beaver} % or try albatross, beaver, crane, ...
  \usefonttheme{default}    % or try default, structurebold, ...
  \setbeamertemplate{navigation symbols}{}
  \setbeamertemplate{caption}[numbered]
}
%\usepackage[utf8]{inputenc}
%\usepackage[T1]{fontenc}
\usepackage{fontspec}
% pour un pdf lisible à l'écran si on ne dispose pas  des fontes cmsuper ou lmodern
\usepackage{pslatex}
\usepackage{graphicx} % Allows including images
\usepackage{booktabs} % Allows the use of \toprule, \midrule and \bottomrule in tables

%----------------------------------------------------------------------------------------
%	TITLE PAGE
%----------------------------------------------------------------------------------------

\title[Machines virtuelles à la demande]{INFO0921: VM à la demande } % The short title appears at the bottom of every slide, the full title is only on the title page

\author{Joffrey.Hérard, Pluot Emmanuel, Derniame Alexandre, Poirier Stéphane} % Your name
\institute[URCA] % Your institution as it will appear on the bottom of every slide, may be shorthand to save space
{
Université de Reims \\ % Your institution for the title page
\medskip
\textit{joffrey.herard@etudiant.univ-reims.fr\\
		emmanuel.pluot@etudiant.univ-reims.fr\\
		alexandre.derniame@etudiant.univ-reims.fr\\
		stephane.poirier@etudiant.univ-reims.fr} 
}
\date{Février 2018} % Date, can be changed to a custom date

\begin{document}

\begin{frame}
\titlepage % Print the title page as the first slide
\end{frame}

\begin{frame}
\frametitle{Sommaire} % Table of contents slide, comment this block out to remove it
\tableofcontents % Throughout your presentation, if you choose to use \section{} and \subsection{} commands, these will automatically be printed on this slide as an overview of your presentation
\end{frame}

%----------------------------------------------------------------------------------------
%	PRESENTATION SLIDES
%----------------------------------------------------------------------------------------

%------------------------------------------------
\section{Introduction} 

\subsection{Définitions} 
\begin{frame}
\frametitle{Définitions}
\begin{block}{toto}
TOTOTOTOTOT
\end{block}
\end{frame}

\subsection{Problématique} 
\begin{frame}
\frametitle{Problématique}
\begin{itemize}
\item abonne toa
\item met la cloch
\item poce bleu
\end{itemize}
\end{frame}
%------------------------------------------------



%----------------------------------------------------------------------------------------

\end{document} 